% ------ Clase de documento ------
\documentclass[a4paper,10pt,oneside]{article}

% ------ Paquetes ------
\usepackage{graphicx}
\usepackage[latin1]{inputenc}
\usepackage[activeacute,spanish]{babel}

% ------ Configuraci�n ------
\title{\textbf{66.20 Organizaci�n de Computadoras\\ Trabajo Pr�ctico 0: \\ Infraestructura b�sica}}

\author{	Burdet Rodrigo, \textit{Padr�n Nro. 93440}\\
            \texttt{ rodrigoburdet@gmail.com}\\\\
            Romani Nazareno, \textit{Padr�n Nro. 83991}                     \\
            \texttt{ nazareno.romani@gmail.com}\\\\
            Martinez Gaston Alberto, \textit{Padr�n Nro. 91383}                     \\
            \texttt{ gaston.martinez.90@gmail.com }\\\\[2.5ex]
            \normalsize{1er. Cuatrimestre de 2014}                       \\
            \normalsize66.20 Organizaci�n de Computadoras\\
            \normalsize{Facultad de Ingenier�a, Universidad de Buenos Aires}            \\
       }
\date{30/03/2014}



% ----- Cuerpo del documento -----
\begin{document}
\maketitle

\thispagestyle{empty}

\newpage

\section{Objetivos}
    Familiarizarse con las herramientas de software que usaremos en los siguientes trabajos, implementando un programa (y su correspondiente documentacion) que resuelva el problema piloto que presentaremos mas abajo.

\section{Resumen}
    En el presente trabajo, se implement� un algoritmo que resuelve la transformaci�n de un conjunto arbitrario de bytes en un conjunto formado por caracteres ASCII y viceversa. Los distintos valores a codificar/decodificar,  son obtenidos a trav�s de los par�metros definidos en el enunciado.
    El programa fue compilado tanto en la m�quina host (sistema operativo linux), como en una m�quina corriendo el sistema operativo NetBSD.

\section{Desarrollo}
	
	\subsection{Paso 1: Configuraci�n de Entorno de Desarrollo}
	El primer paso fue configurar el entorno de desarrollo, de acuerdo a la gu�a facilitada por la c�tedra. \\
	Trabajamos con la distribuci�n de linux basada en Debian y con el GxEmul proporcionado por la c�tedra, 
	el cual tiene ya configurado NetBSD.
	
	\subsection{Paso 2: Implementaci�n del programa}
	El programa debe ejecutarse por l�nea de comando y la salida del mismo depender� del valor de los argumentos con los que se lo haya invocado.
	\subsubsection{Ingreso de par�metros}
		El formato para invocar al programa es el siguiente:
		\begin{center}
			./tp0 [OPTIONS]
		\end{center}
	Los par�metros v�lidos que puede recibir el programa son los siguientes: 
	

		\begin{tabbing}
		 \textbf{-e , --encode} 	(Encodes to Base64). \\
		\textbf{-d , --decode} (Decodes from Base64).\\
		 \textbf{-i , --input file} 		(Reads from file or stdin).\\
		\textbf{-o , --output file} 		(Writes to file or stdout).\\
		\textbf{-v , --version} 		(Show version string).\\
		\textbf{-h , --help} 		(Print this message and quit).\\
		\end{tabbing}
	En caso de recibir un par�metro diferente a alguno de los listados anteriormente, la salida del programa ser� la misma que la de la opci�n -h, para despejar posibles dudas de uso.  VER VER
		
	\subsubsection{Interpretaci�n de par�metros}
		Para parsear los par�metros se usaron las funciones definidas en arg\_parse.h. Se puede conocer m�s en detalle el funcionamiento de las mismas, a trav�s de la documentaci�n incluida en dicho archivo. 
		Estas funciones permiten recoger los par�metros de entrada del programa y ejecutar la funcionalidad correspondiente. Estas son compatibles con NetBSD.
			
	\section{C�digo para compilar el programa con gcc}
	
		Para poder compilar el proyecto, se debe abrir una terminal linux dentro del directorio donde se encuentra el c�digo 
		fuente escrito en C, y utilizar el siguiente comando:
		\begin{center}
			\texttt{gcc -o tp0 tp0.c }
		\end{center}

	Esto generara un archivo ejecutable, llamado tp0, con el que luego realizaremos las pruebas.\newline
	Para generar el c�gido en MIPS32, generado por el compilador, vamos a usar el siguiente comando:
		\begin{center}
			gcc -march=mips32 -S -c tp0.c -o tp0.S
		\end{center}
		
	
\section{Corridas de prueba y Mediciones}

	En las figuras que siguen a continuaci�n se muestran los comandos utilizados para ejecutar el programa y se puede apreciar en los gr�ficos los resultados de las diferentes pruebas que realizamos.  \newpage
	
	\begin{figure}[!htp]
		\begin{center}
			\includegraphics[width=0.50\textwidth]{corrida1.png}
		\end{center}
		\caption{Corrida 1} \label{Corrida 1}
	\end{figure}

	\begin{figure}[!htp]
		\begin{center}
			\includegraphics[width=0.50\textwidth]{corrida2.png}
		\end{center}
		\caption{Corrida 2} \label{Corrida 2}
	\end{figure}

	\begin{figure}[!htp]
		\begin{center}
			\includegraphics[width=0.50\textwidth]{corrida3.png}
		\end{center}
		\caption{Corrida 3} \label{Corrida 3}
	\end{figure}


\section{Conclusiones}	
	Como se enuncia en el objetivo de este trabajo pr�ctico, aprendimos a instalar y manejar el GxEmul, a realizar transferencias de archivos en linux, as� como tambi�n compilar y ejecutar programas en el NetBSD. Por otro lado,  aprendimos a manejar y escribir informes en \LaTeX{}.
	De este modo, estamos preparados para que en los pr�ximos trabajos pr�cticos, nos aboquemos directamente al desarrollo de los mismos.

\end{document}

